% This template found at: https://tex.stackexchange.com/questions/42602/software-requirements-specification-with-latex
% It was a relatively simple template provided by SE user "Yiannis Lazarides"
% whose page is here: https://tex.stackexchange.com/users/963/yiannis-lazarides

%The scrreprt is part of the KOMA-script bundle of packages that does some 
%Fancy-schmancy tweaks to the typography of the resulting document. You know, 
%the kind of stuff that editors and designers love!
\documentclass{scrreprt}

%apparently something is deprecated, but using this package fixes that problem.
\usepackage{scrhack} 

%For code listings
\usepackage{listings}

%I honestly don't know what this is here for. Apparently it addresses some 
%weird edge case having to do with underscores and hyphenation of words. We 
%Could probably do without this package, but perhaps it's required by another 
%package.
\usepackage{underscore}

%How to handle hyperlinks.
\usepackage[bookmarks=true]{hyperref}
\hypersetup{
    bookmarks=true,    % show bookmarks bar?
    pdftitle={Software Requirement Specification},    % title
    pdfauthor={Team Moon Moon},                     % author
    pdfsubject={TeX and LaTeX},                        % subject of the document
    pdfkeywords={TeX, LaTeX, graphics, images}, % list of keywords
    colorlinks=true,       % false: boxed links; true: colored links
    linkcolor=blue,       % color of internal links
    citecolor=black,       % color of links to bibliography
    filecolor=black,        % color of file links
    urlcolor=purple,        % color of external links
    linktoc=all            % all makes whole line a link. "page" -> only page#
}%

\def\myversion{0.1.0 }
\title{%
\flushright
\rule{16cm}{5pt}\vskip1cm
\Huge{SOFTWARE REQUIREMENTS\\ SPECIFICATION}\\
\vspace{2cm}
for\\
\vspace{2cm}
Online Postage Ordering System\\
\vspace{2cm}
\LARGE{Release 0.1.0\\}
\vspace{2cm}
\LARGE{Version \myversion approved\\}
\vspace{2cm}
Prepared by Team Moon Moon\\
\vfill
\rule{16cm}{5pt}
}

\date{}

\usepackage{hyperref}
\author{Team Moon Moon}

\begin{document}
\maketitle
\tableofcontents

\chapter*{Revision History}

\begin{enumerate}
\item Team Moon Moon Needs No Revisions.
\item Team Moon Moon Never Makes No Mistakes.
\item Correction. Team Moon Moon ``fails fast.'' Isn't that the new 
buzzword or buzzphrase?
\item The skeleton has been updated to reflect the actual requirements 
provided by Dr. Seaman.
\end{enumerate}

\chapter{Introduction}

Team Moon Moon has been tasked with developing a SRS for their Software 
Engineering class at Texas State University. If Team Moon Moon doesn't do 
this project, Team Moon Moon will fail D: Team Moon Moon wants to 
graduate already and doesn't feel like taking this class again.

\section{Purpose}

The purpose of this project is to get our C. Because C's get degrees!

\section{Project Scope and Product Features}

BLAH BLAH BLAH

\section{Defintions, Acronyms, and Abbreviations}

What is the airspeed velocity of an unladen swallow?

\section{References}

OMG OMG OMG

\section{Overview}

Aren't ``Introduction'' and ``Overview'' kinda the same thing? Just sayin'...

\chapter{Overall Description}

Many description. Much adjective. Wow.

\section{Product Perspective}

Such perspective. So edgy.

\section{Product Functions}

BRO DO YOU EVEN FUNCTION?

\subsection{Create An Account}

Lol please shoot me now nobody does software development like this anymore.

\subsection{Print Postage}

The Print Postage function shall allow the user to print postage that has been 
purchased, either to a file or directly to a printer.

\subsection{Buy Postage}

TO THE MOON!!!

\subsection{List Postage Transactions}

herpderp

\subsection{Get Postage Balance}

has anyone ever been so far as even decided.

\subsection{Calculate Postage}

A Stamp? What's a stamp, grampa?

\subsection{Track/Confirm Package}

Here, you can incessantly check on the state of your package inbetween meals 
and trips to the loo! Never be out of the loop again. You really needed to 
know that your package left the seller's facility.

\subsection{Login}

I can't let you do that Dave.

I can't let you do that, Dave.

Commas.

\subsection{Refund Request}

It's been denied.

\subsection{Validate Address}

Validating addresses is actually a royal PITA. Just let Google Maps API 
handle it.

\section{User Characteristics}

Our users are in the 19-29 demographic. They love iProducts of all sorts and 
can't wait to buy postage stamps with our hip new website!

\section{Constraints}

Batman has no constraints.

\section{Assumptions and Dependencies}

I'm starting to run out of steam here guys.

\chapter{Specific Requirements}

As opposed to general requirements.

\section{External Interface Requirements}

aryweguiahergearpguhestrgperasguheasrgpewuhr

\subsection{System Interfaces}

This product will interface with a Gibson. You know? Like, supercomputers 
they use to do physics and search for oil? Wouldn't you love to get one of 
those Gibsons?

\subsection{User Interfaces}

CLI fo LYFE. Firefox is for nubs. Real haxors use lynx!

\subsection{Hardware Interfaces}

awrygaeghsrethsegwearfawfewaf. I feel better now.

\subsection{Software Interfaces}

gbsetghbsetgesrgfawefawefwaefawfawefawefwaf. Much better.

\subsection{Communication Interfaces}

Does the internet constitute a communication, software, or hardware 
interface? All of the above? All of the others? Some? Who knows.

\section{Functional Requirements}

Ok. Now I'm really running low on ideas.

\subsection{Create an Account}

The system shall spam you. Violently.

PLEASE GIVE US YOUR EMAIL SO WE CAN SPAM YOU! I mean... 
OFFER YOU EXCLUSIVE PROMOTIONS ON INSIDER DEALS!

\subsection{Print Postage}

The system shall...

\begin{enumerate}
\item First, ensure...
\begin{enumerate}
\item that there is a valid, logged-in user requesting the postage.
\item that the user is authorized to print the postage.
\item that the user has sufficient credit to print the requested quantity of 
postage.
\item that the user's printer is available and functional.
\end{enumerate}
\item Then, the system shall prompt the user to confirm the printing of the 
postage, showing current and subsequent account balances as necessary.
\item Then, once confirmation from the user is obtained, the system shall 
generate the postage on a special page, devoid of any other content, that the 
user may print.
\item Finally, the system shall, upon detection of the completed print job, 
return the user to the ``account status'' page, where the new balance is 
displayed. The system shall also update the new balance in the user's account 
and add the job to the user's usage-history.
\end{enumerate}

\subsection{Buy Postage}

The system shall take your money and run. Maniacally.

\subsection{List Postage Transactions}

The system shall list all the times it serviced your account. Suggestively.

\subsection{Get Postage Balance}

The system shall present your balance. And laugh.

\subsection{Calculate Postage}

The system shall calculate how much postage is required to mail your package. 
Too much.

\subsection{Track/Confirm Package}

The system shall give you your tracking number. A day later.

\subsection{Login}

The system shall let you in upon uttering the magic word. Please.

\subsection{Refund Request}

The system shall take your request and pipe it to /dev/null. Cheerfully.

\subsection{Validate Address}

The system shall throw its hands in the air and just flip a coin to determine 
whether the address is valid.

\section{Performance Requirements}

Don't fail.

\section{Logical Structure of the Data}

That would imply ``logic'' and ``structure,'' neither of which are 
currently present. The logic is back-ordered and the structure was 
never there to begin with.

In any case, this is where the main diagram goes.

\section{Design Constraints}

Why would we put design constraints all the way at the bottom? You'd think 
the design constraints might illuminate pretty much all of the subsequent 
design decisions that were made and expounded upon in the entire SRS...

\section{Software System Attributes}

Is it at-tri-bute? Or a-trib-ute? Because one is a noun and one is a verb. 
Yeah, it's kinda the English language's way of saying FUCK YOU to foreigners.

% add other chapters and sections to suit
\end{document}
