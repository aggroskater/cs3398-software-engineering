% This template found at: https://tex.stackexchange.com/questions/42602/software-requirements-specification-with-latex
% It was a relatively simple template provided by SE user "Yiannis Lazarides"
% whose page is here: https://tex.stackexchange.com/users/963/yiannis-lazarides

%The scrreprt is part of the KOMA-script bundle of packages that does some 
%Fancy-schmancy tweaks to the typography of the resulting document. You know, 
%the kind of stuff that editors and designers love!
\documentclass{scrreprt}

%apparently something is deprecated, but using this package fixes that problem.
\usepackage{scrhack} 

%For code listings
\usepackage{listings}

%I honestly don't know what this is here for. Apparently it addresses some 
%weird edge case having to do with underscores and hyphenation of words. We 
%Could probably do without this package, but perhaps it's required by another 
%package.
\usepackage{underscore}

%How to handle hyperlinks.
\usepackage[bookmarks=true]{hyperref}
\hypersetup{
    bookmarks=true,    % show bookmarks bar?
    pdftitle={Software Requirement Specification},    % title
    pdfauthor={Team Moon Moon},                     % author
    pdfsubject={TeX and LaTeX},                        % subject of the document
    pdfkeywords={TeX, LaTeX, graphics, images}, % list of keywords
    colorlinks=true,       % false: boxed links; true: colored links
    linkcolor=blue,       % color of internal links
    citecolor=black,       % color of links to bibliography
    filecolor=black,        % color of file links
    urlcolor=purple,        % color of external links
    linktoc=all            % all makes whole line a link. "page" -> only page#
}%

\def\myversion{0.1.0 }
\title{%
\flushright
\rule{16cm}{5pt}\vskip1cm
\Huge{SOFTWARE REQUIREMENTS\\ SPECIFICATION}\\
\vspace{2cm}
for\\
\vspace{2cm}
Online Postage Ordering System\\
\vspace{2cm}
\LARGE{Release 0.1.0\\}
\vspace{2cm}
\LARGE{Version \myversion approved\\}
\vspace{2cm}
Prepared by Team Moon Moon\\
\vfill
\rule{16cm}{5pt}
}

\date{}

\usepackage{hyperref}
\author{Team Moon Moon}

\begin{document}
\maketitle
\tableofcontents

\chapter*{Revision History}

\begin{enumerate}
\item Team Moon Moon Needs No Revisions.
\item Team Moon Moon Never Makes No Mistakes.
\item Correction. Team Moon Moon ``fails fast.'' Isn't that the new 
buzzword or buzzphrase?
\item The skeleton has been updated to reflect the actual requirements 
provided by Dr. Seaman.
\item It's all really coming together now.
\end{enumerate}

\chapter{Introduction}

IN-PROGRESS

\section{Purpose}

The purpose of this document is to present a detailed description of the
specifications and the requirements for a postage printing website. It will
explain the purpose and features of the system, the interfaces of the system,
what the system will do, the constraints under which it must operate, and how
the system will react to external stimuli. This document is intended for both
the stakeholders and the developers of the system.

\section{Project Scope and Product Features}

The objective of this project is to create and implement a website for printing
mailing labels containing USPS postage. The website will be used primarily by
registered users. The website will allow users to: 

\begin{itemize}
\item Create and maintain individual secured accounts
\item Buy/print postage labels
\item List postage purchase history
\item Check account balance
\item Calculate postage rates
\item Track packages
\item Request refunds
\item Check if an address is valid
\end{itemize}

\section{Defintions, Acronyms, and Abbreviations}

IN-PROGRESS

\section{References}

IEEE. IEEE Std 830-1998 IEEE Recommended Practice for Software Requirements
Specifications. IEEE Computer Society, 1998.

\section{Overview}

The next section, the Overall Description, of this document gives an overview
of the functionality of the product. It describes the informal requirements and
is used to establish a context for the technical requirements specification in
the next chapter. 

The third section, Requirements Specification, of this document is written
primarily for the developers and describes, in technical terms, the details of
the functionality of the product. 

\chapter{Overall Description}

IN-PROGRESS.

\section{Product Perspective}

IN-PROGRESS.

\section{Product Functions}

IN-PROGRESS.

\subsection{Create An Account}

The Create Account function shall allow a user to create a secure account. The
account will track the user’s full name, mailing address, email address, credit
card information, username and password, and the purchase history for up to two
years.

\subsection{Print Postage}

The Print Postage function shall allow the user to print postage that has been 
purchased, either to a file or directly to a printer.

\subsection{Buy Postage}

IN-PROGRESS.

\subsection{List Postage Transactions}

IN-PROGRESS.

\subsection{Get Postage Balance}

The Get Postage Balance function shall allow the user to obtain their current
amount in their postage buying account.

\subsection{Calculate Postage}

The Calculate Postage function shall allow the user to enter a name, address,
package type, mail class, and weight. The user shall then be able to see a
calculated postage amount to ship their package.

\subsection{Track/Confirm Package}

The track/confirm package function shall allow users to enter in a tracking
number to see the status of their package.

\subsection{Login}

The login function shall allow account members to enter their username and
password.  When verified with the system database, users will be able to access
restricted functions.[TBD] The login function shall provide users an option to
reset their password. 

\subsection{Refund Request}

IN-PROGRESS.

\subsection{Validate Address}

IN-PROGRESS.

\subsection{Logout}

The logout function shall allow the account user to exit out of their account
for security.

\section{User Characteristics}

IN-PROGRESS.

\section{Constraints}

IN-PROGRESS.

\section{Assumptions and Dependencies}

IN-PROGRESS.

\chapter{Specific Requirements}

IN-PROGRESS.

\section{External Interface Requirements}

IN-PROGRESS.

\subsection{System Interfaces}

IN-PROGRESS.

\subsection{User Interfaces}

IN-PROGRESS.

\subsection{Hardware Interfaces}

IN-PROGRESS.

\subsection{Software Interfaces}

IN-PROGRESS.

\subsection{Communication Interfaces}

IN-PROGRESS.

\section{Functional Requirements}

IN-PROGRESS.

\subsection{Create an Account}

IN-PROGRESS.

\subsection{Print Postage}

The system shall...

\begin{enumerate}
\item First, ensure...
\begin{enumerate}
\item that there is a valid, logged-in user requesting the postage.
\item that the user is authorized to print the postage.
\item that the user has sufficient credit to print the requested quantity of 
postage.
\item that the user's printer is available and functional.
\end{enumerate}
\item Then, the system shall prompt the user to confirm the printing of the 
postage, showing current and subsequent account balances as necessary.
\item Then, once confirmation from the user is obtained, the system shall 
generate the postage on a special page, devoid of any other content, that the 
user may print.
\item Finally, the system shall, upon detection of the completed print job, 
return the user to the ``account status'' page, where the new balance is 
displayed. The system shall also update the new balance in the user's account 
and add the job to the user's usage-history.
\end{enumerate}

\subsection{Buy Postage}

IN-PROGRESS.

\subsection{List Postage Transactions}

IN-PROGRESS.

\subsection{Get Postage Balance}

The system shall...

\begin{enumerate}
\item First, ensure...
\begin{enumerate}
\item that there is a valid, logged-in user requesting the postage balance.
\item that the user is authorized to view the postage balance.
\end{enumerate}
\item Then, the system shall find user account, and display postage balance 
in order for the user to see.
\end{enumerate}

\subsection{Calculate Postage}

The system shall...

\begin{enumerate}
\item First, ensure that there is a valid, logged-in user requesting the
postage balance.  
\item Then, the system shall provide a method for the user to enter a
Name,Address,Packaging type, mail class, and weight.
\item Then, the system shall calculate estimated shipping cost for the users
package.
\item Finally, the system shall, upon completion of calculation, display the
estimated shipping costs.  
\end{enumerate}

\subsection{Track/Confirm Package}

The system shall...

\begin{enumerate}
\item allow the user to enter in a tracking number.
\item check the tracking number with the system database.
\begin{enumerate}
\item if valid, the system shall display the current tracking information for
the user to access.
\item if invalid, the system shall inform the user that the tracking number is
not in the system database. The user may enter a tracking number or the user
may cancel. 
\end{enumerate}
\end{enumerate}

\subsection{Login}

The system shall...

\begin{enumerate}
\item require a username and password from the user.
\item verify the username and password with the system database.
\begin{enumerate}
\item If valid, the user will be granted access to the following functions:
[TBD]
\item If the username is not located within the system database, the system shall
inform the user that their username is invalid. The user may enter a different
username or the user may cancel.
\item If password is not valid for the username entered, the system shall inform the
user that the password is invalid. The user many enter a different password or
the user may cancel
\end{enumerate}
\item allow the user to request a password from the system to be emailed to the
registered user.
\begin{enumerate}
\item The user can request a password from the system, provided the user gives
the system a valid username from the system database.
\begin{enumerate}
\item If the username is not within the database the system shall inform the
user that their username is invalid. The user may enter a different username of
the user may cancel.
\end{enumerate}
\item The system shall randomize a password and send to the users accounts
email in the system database.
\end{enumerate}
\end{enumerate}

\subsection{Refund Request}

IN-PROGRESS.

\subsection{Validate Address}

IN-PROGRESS.

\subsection{Logout}

The system shall...

\begin{enumerate}
\item immediately disable access to functions that require a registered user.
\end{enumerate}

\section{Performance Requirements}

IN-PROGRESS.

\section{Logical Structure of the Data}

IN-PROGRESS. Assigned to Preston Maness (me). I'm still working on a good 
way to diagram everything.

\section{Design Constraints}

Constraints for the postage printing website include

\begin{itemize}
\item Able to support PC, Mac platforms.
\item System logs out user after ten minutes of inactivity.
\item System supports all web browsers.
\end{itemize}

\section{Software System Attributes}

IN-PROGRESS.

% add other chapters and sections to suit
\end{document}
